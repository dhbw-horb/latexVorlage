%!TEX root = ../dokumentation.tex

%
% Nahezu alle Einstellungen koennen hier getaetigt werden
%

\documentclass[%
	pdftex,
	oneside,			% Einseitiger Druck.
	12pt,				% Schriftgroesse
	parskip=half,		% Halbe Zeile Abstand zwischen Absätzen.
	headsepline,		% Linie nach Kopfzeile.
	footsepline,		% Linie vor Fusszeile.
	abstracton,			% Abstract Überschriften
	listof=totoc,
	toc=bibliography,
]{scrreprt}

% Einstellungen laden
\usepackage{xstring}
\usepackage[utf8]{inputenc}
\usepackage[T1]{fontenc}

\newcommand{\einstellung}[1]{%
  \expandafter\newcommand\csname #1\endcsname{}
  \expandafter\newcommand\csname setze#1\endcsname[1]{\expandafter\renewcommand\csname#1\endcsname{##1}}
}
\newcommand{\langstr}[1]{\einstellung{lang#1}}

\einstellung{matrikelnr}
\einstellung{titel}
\einstellung{kurs}
\einstellung{datumAbgabe}
\einstellung{firma}
\einstellung{firmenort}
\einstellung{abgabeort}
\einstellung{abschluss}
\einstellung{studiengang}
\einstellung{dhbw}
\einstellung{betreuer}
\einstellung{gutachter}
\einstellung{zeitraum}
\einstellung{arbeit}
\einstellung{autor}
\einstellung{sprache}
\einstellung{schriftart}
\einstellung{seitenrand}
\einstellung{kapitelabstand}
\einstellung{spaltenabstand}
\einstellung{zeilenabstand}
\einstellung{zitierstil}
 % verfügbare Einstellungen
\setzemartrikelnr{1234510}
\setzekurs{ABC2008DE}
\setzetitel{In der Regel haben wir einen zweizeiligen Bachelorthesistitel}
\setzedatumAbgabe{August 2011}
\setzefirma{Firma GmbH}
\setzefirmenort{Firmenort}
\setzeabgabeort{Abgabeort}
\setzeabschluss{Bachelor of Engineering}
\setzestudiengang{Studienganges Vorderasiatische Archäologie}
\setzedhbw{Karlsruhe}
\setzebetreuer{Dipl.-Ing. (FH) Peter Pan}
\setzegutachter{Dr.\ Silvana Koch-Mehrin}
\setzezeitraum{12 Wochen}
\setzearbeit{Bachelorarbeit}
\setzeautor{Vorname Nachname}
\setzesprache{de} % oder en
 % lese Einstellungen

\newcommand{\iflang}[2]{%
  \IfStrEq{\sprache}{#1}{#2}{}
}

\langstr{abkverz}
\langstr{glossar}
\langstr{deckblattabschlusshinleitung}
\langstr{artikelstudiengang}
\langstr{studiengang}
\langstr{anderdh}
\langstr{von}
\langstr{dbbearbeitungszeit}
\langstr{dbmatriknr}
\langstr{dbkurs}
\langstr{dbfirma}
\langstr{dbbetreuer}
\langstr{dbgutachter}
\langstr{sperrvermerk}
\langstr{erklaerung}
\langstr{abstract}
 % verfügbare Strings
\input{lang/\sprache} % Übersetzung einlesen

\usepackage[english, ngerman]{babel}
\iflang{de}{\selectlanguage{ngerman}} % Paket babel benutzt neue deutsche Rechtschreibung
\iflang{en}{\selectlanguage{english}} % Paket babel benutzt Englisch


%%%%%%% Package Includes %%%%%%%

\usepackage[margin=1in]{geometry}
\usepackage[activate]{microtype} %Zeilenumbruch und mehr
\usepackage[onehalfspacing]{setspace}
\usepackage{makeidx}
\usepackage[autostyle=true,german=quotes]{csquotes}
\usepackage{longtable}
\usepackage{graphicx}
\usepackage{xcolor} 	%xcolor für HTML-Notation
\usepackage{float}
\usepackage{array}
\usepackage{calc}		%zum Rechnen (Bildtabelle in Deckblatt)
\usepackage[right]{eurosym}
\usepackage{wrapfig}
\usepackage{pgffor} % für automatische Kapiteldateieinbindung
\usepackage[perpage, hang, multiple, stable]{footmisc}
\usepackage[printonlyused,footnote]{acronym}

\usepackage{listings}

% eine Kommentarumgebung "k" (Handhabe mit \begin{k}<Kommentartext>\end{k},
% Kommentare werden rot gedruckt). Wird \% vor excludecomment{k} entfernt,
% werden keine Kommentare mehr gedruckt.
\usepackage{comment}
\specialcomment{k}{\begingroup\color{red}}{\endgroup}
%\excludecomment{k}


%%%%%% Configuration %%%%%

%% Anwenden der Einstellungen

\usepackage{\schriftart}
\ladefarben{}

% Titel, Autor und Datum
\title{\titel}
\author{\autor}
\date{\datum}

% PDF Einstellungen
\usepackage[%
	pdftitle={\titel},
	pdfauthor={\autor},
	pdfsubject={\arbeit},
	pdfcreator={pdflatex, LaTeX with KOMA-Script},
	pdfpagemode=UseOutlines, 		% Beim Oeffnen Inhaltsverzeichnis anzeigen
	pdfdisplaydoctitle=true, 		% Dokumenttitel statt Dateiname anzeigen.
	pdflang={\sprache}, 			% Sprache des Dokuments.
]{hyperref}

% (Farb-)einstellungen für die Links im PDF
\hypersetup{%
	colorlinks=true, 		% Aktivieren von farbigen Links im Dokument
	linkcolor=LinkColor, 	% Farbe festlegen
	citecolor=LinkColor,
	filecolor=LinkColor,
	menucolor=LinkColor,
	urlcolor=LinkColor,
	linktocpage=true, 		% Nicht der Text sondern die Seitenzahlen in Verzeichnissen klickbar
	bookmarksnumbered=true 	% Überschriftsnummerierung im PDF Inhalt anzeigen.
}
% Workaround um Fehler in Hyperref, muss hier stehen bleiben
\usepackage{bookmark} %nur ein latex-Durchlauf für die Aktualisierung von Verzeichnissen nötig

% Schriftart in Captions etwas kleiner
\addtokomafont{caption}{\small}

% Literaturverweise (sowohl deutsch als auch englisch)
\iflang{de}{%
\usepackage[
	backend=biber,		% empfohlen. Falls biber Probleme macht: bibtex
	bibwarn=true,
	bibencoding=utf8,	% wenn .bib in utf8, sonst ascii
	sortlocale=de_DE,
	style=\zitierstil,
]{biblatex}
}
\iflang{en}{%
\usepackage[
	backend=biber,		% empfohlen. Falls biber Probleme macht: bibtex
	bibwarn=true,
	bibencoding=utf8,	% wenn .bib in utf8, sonst ascii
	sortlocale=en_US,
	style=\zitierstil,
]{biblatex}
}

\ladeliteratur{}

% Glossar
\usepackage[nonumberlist,toc]{glossaries}

%%%%%% Additional settings %%%%%%

% Hurenkinder und Schusterjungen verhindern
% http://projekte.dante.de/DanteFAQ/Silbentrennung
\clubpenalty=10000
\widowpenalty=10000
\displaywidowpenalty=10000

% Bildpfad
\graphicspath{{images/}}

% Einige häufig verwendete Sprachen
\lstloadlanguages{PHP,Python,Java,C,C++,bash}
\listingsettings{}
% Umbennung des Listings
\renewcommand\lstlistingname{\langlistingname}
\renewcommand\lstlistlistingname{\langlistlistingname}
\def\lstlistingautorefname{\langlistingautorefname}

% Abstände in Tabellen
\setlength{\tabcolsep}{\spaltenabstand}
\renewcommand{\arraystretch}{\zeilenabstand}
