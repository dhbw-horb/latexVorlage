%!TEX root = ../dokumentation.tex

\pagestyle{empty}

\iflang{de}{%
% Dieser deutsche Teil wird nur angezeigt, wenn die Sprache auf Deutsch eingestellt ist.
%Wenn das Abstract auf Deutsch sein soll die Zeile mit  \selectlanguage{english} auskommentieren
\selectlanguage{english}
\renewcommand{\abstractname}{Abstract}
\begin{abstract}
In englischer Sprache! Max. eine Seite und
eigenständig verständlich! 

Ein Abstract ist eine kurze und aussagekräftige Beschreibung des Inhalts der Arbeit. Der Umfang umfasst in
der Regel 200 bis 250 Wörter und beinhaltet die Fragestellung der Arbeit, die methodische Vorgehensweise
sowie die Hauptergebnisse der Arbeit. 

Quelle: \url{http://www.dhbw.de/fileadmin/user/public/Dokumente/Portal/Richtlinien_Praxismodule_Studien_und_Bachelorarbeiten_JG2011ff.pdf},Seite 22, Abgerufen 15.02.2015
\end{abstract}
\selectlanguage{ngerman}
}


\iflang{en}{%
% Dieser englische Teil wird nur angezeigt, wenn die Sprache auf Englisch eingestellt ist.
\begin{abstract}
An abstract is a brief summary of a research article, thesis, review,
conference proceeding or any in-depth analysis of a particular subject
or discipline, and is often used to help the reader quickly ascertain
the paper's purpose. When used, an abstract always appears at the
beginning of a manuscript, acting as the point-of-entry for any given
scientific paper or patent application. Abstracting and indexing
services for various academic disciplines are aimed at compiling a
body of literature for that particular subject.

The terms précis or synopsis are used in some publications to refer to
the same thing that other publications might call an "abstract". In
management reports, an executive summary usually contains more
information (and often more sensitive information) than the abstract
does.

Quelle: \url{http://en.wikipedia.org/wiki/Abstract_(summary)}

\end{abstract}
}