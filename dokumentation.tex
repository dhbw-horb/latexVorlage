%%**************************************************************
%% Vorlage fuer Bachelorarbeiten (o.ä.) der DHBW
%%
%% Autor: Tobias Dreher, Yves Fischer
%% Datum: 06.07.2011
%%**************************************************************

\newcommand{\pdftitel}{In der Regel haben wir einen zweizeiligen Bachelorthesistitel}
\newcommand{\autor}{Vorname Nachname}
\newcommand{\arbeit}{Bachelorarbeit}

%!TEX root = ../dokumentation.tex

%
% Nahezu alle Einstellungen koennen hier getaetigt werden
%

\documentclass[%
	pdftex,
	oneside,			% Einseitiger Druck.
	12pt,				% Schriftgroesse
	parskip=half,		% Halbe Zeile Abstand zwischen Absätzen.
	headsepline,		% Linie nach Kopfzeile.
	footsepline,		% Linie vor Fusszeile.
	abstracton,			% Abstract Überschriften
	listof=totoc,
	toc=bibliography,
]{scrreprt}

% Einstellungen laden
\usepackage{xstring}
\usepackage[utf8]{inputenc}
\usepackage[T1]{fontenc}

\newcommand{\einstellung}[1]{%
  \expandafter\newcommand\csname #1\endcsname{}
  \expandafter\newcommand\csname setze#1\endcsname[1]{\expandafter\renewcommand\csname#1\endcsname{##1}}
}
\newcommand{\langstr}[1]{\einstellung{lang#1}}

\einstellung{matrikelnr}
\einstellung{titel}
\einstellung{kurs}
\einstellung{datumAbgabe}
\einstellung{firma}
\einstellung{firmenort}
\einstellung{abgabeort}
\einstellung{abschluss}
\einstellung{studiengang}
\einstellung{dhbw}
\einstellung{betreuer}
\einstellung{gutachter}
\einstellung{zeitraum}
\einstellung{arbeit}
\einstellung{autor}
\einstellung{sprache}
\einstellung{schriftart}
\einstellung{seitenrand}
\einstellung{kapitelabstand}
\einstellung{spaltenabstand}
\einstellung{zeilenabstand}
\einstellung{zitierstil}
 % verfügbare Einstellungen
\setzemartrikelnr{1234510}
\setzekurs{ABC2008DE}
\setzetitel{In der Regel haben wir einen zweizeiligen Bachelorthesistitel}
\setzedatumAbgabe{August 2011}
\setzefirma{Firma GmbH}
\setzefirmenort{Firmenort}
\setzeabgabeort{Abgabeort}
\setzeabschluss{Bachelor of Engineering}
\setzestudiengang{Studienganges Vorderasiatische Archäologie}
\setzedhbw{Karlsruhe}
\setzebetreuer{Dipl.-Ing. (FH) Peter Pan}
\setzegutachter{Dr.\ Silvana Koch-Mehrin}
\setzezeitraum{12 Wochen}
\setzearbeit{Bachelorarbeit}
\setzeautor{Vorname Nachname}
\setzesprache{de} % oder en
 % lese Einstellungen

\newcommand{\iflang}[2]{%
  \IfStrEq{\sprache}{#1}{#2}{}
}

\langstr{abkverz}
\langstr{glossar}
\langstr{deckblattabschlusshinleitung}
\langstr{artikelstudiengang}
\langstr{studiengang}
\langstr{anderdh}
\langstr{von}
\langstr{dbbearbeitungszeit}
\langstr{dbmatriknr}
\langstr{dbkurs}
\langstr{dbfirma}
\langstr{dbbetreuer}
\langstr{dbgutachter}
\langstr{sperrvermerk}
\langstr{erklaerung}
\langstr{abstract}
 % verfügbare Strings
\input{lang/\sprache} % Übersetzung einlesen

\usepackage[english, ngerman]{babel}
\iflang{de}{\selectlanguage{ngerman}} % Paket babel benutzt neue deutsche Rechtschreibung
\iflang{en}{\selectlanguage{english}} % Paket babel benutzt Englisch


%%%%%%% Package Includes %%%%%%%

\usepackage[margin=1in]{geometry}
\usepackage[activate]{microtype} %Zeilenumbruch und mehr
\usepackage[onehalfspacing]{setspace}
\usepackage{makeidx}
\usepackage[autostyle=true,german=quotes]{csquotes}
\usepackage{longtable}
\usepackage{graphicx}
\usepackage{xcolor} 	%xcolor für HTML-Notation
\usepackage{float}
\usepackage{array}
\usepackage{calc}		%zum Rechnen (Bildtabelle in Deckblatt)
\usepackage[right]{eurosym}
\usepackage{wrapfig}
\usepackage{pgffor} % für automatische Kapiteldateieinbindung
\usepackage[perpage, hang, multiple, stable]{footmisc}
\usepackage[printonlyused,footnote]{acronym}

\usepackage{listings}

% eine Kommentarumgebung "k" (Handhabe mit \begin{k}<Kommentartext>\end{k},
% Kommentare werden rot gedruckt). Wird \% vor excludecomment{k} entfernt,
% werden keine Kommentare mehr gedruckt.
\usepackage{comment}
\specialcomment{k}{\begingroup\color{red}}{\endgroup}
%\excludecomment{k}


%%%%%% Configuration %%%%%

%% Anwenden der Einstellungen

\usepackage{\schriftart}
\ladefarben{}

% Titel, Autor und Datum
\title{\titel}
\author{\autor}
\date{\datum}

% PDF Einstellungen
\usepackage[%
	pdftitle={\titel},
	pdfauthor={\autor},
	pdfsubject={\arbeit},
	pdfcreator={pdflatex, LaTeX with KOMA-Script},
	pdfpagemode=UseOutlines, 		% Beim Oeffnen Inhaltsverzeichnis anzeigen
	pdfdisplaydoctitle=true, 		% Dokumenttitel statt Dateiname anzeigen.
	pdflang={\sprache}, 			% Sprache des Dokuments.
]{hyperref}

% (Farb-)einstellungen für die Links im PDF
\hypersetup{%
	colorlinks=true, 		% Aktivieren von farbigen Links im Dokument
	linkcolor=LinkColor, 	% Farbe festlegen
	citecolor=LinkColor,
	filecolor=LinkColor,
	menucolor=LinkColor,
	urlcolor=LinkColor,
	linktocpage=true, 		% Nicht der Text sondern die Seitenzahlen in Verzeichnissen klickbar
	bookmarksnumbered=true 	% Überschriftsnummerierung im PDF Inhalt anzeigen.
}
% Workaround um Fehler in Hyperref, muss hier stehen bleiben
\usepackage{bookmark} %nur ein latex-Durchlauf für die Aktualisierung von Verzeichnissen nötig

% Schriftart in Captions etwas kleiner
\addtokomafont{caption}{\small}

% Literaturverweise (sowohl deutsch als auch englisch)
\iflang{de}{%
\usepackage[
	backend=biber,		% empfohlen. Falls biber Probleme macht: bibtex
	bibwarn=true,
	bibencoding=utf8,	% wenn .bib in utf8, sonst ascii
	sortlocale=de_DE,
	style=\zitierstil,
]{biblatex}
}
\iflang{en}{%
\usepackage[
	backend=biber,		% empfohlen. Falls biber Probleme macht: bibtex
	bibwarn=true,
	bibencoding=utf8,	% wenn .bib in utf8, sonst ascii
	sortlocale=en_US,
	style=\zitierstil,
]{biblatex}
}

\ladeliteratur{}

% Glossar
\usepackage[nonumberlist,toc]{glossaries}

%%%%%% Additional settings %%%%%%

% Hurenkinder und Schusterjungen verhindern
% http://projekte.dante.de/DanteFAQ/Silbentrennung
\clubpenalty=10000
\widowpenalty=10000
\displaywidowpenalty=10000

% Bildpfad
\graphicspath{{images/}}

% Einige häufig verwendete Sprachen
\lstloadlanguages{PHP,Python,Java,C,C++,bash}
\listingsettings{}
% Umbennung des Listings
\renewcommand\lstlistingname{\langlistingname}
\renewcommand\lstlistlistingname{\langlistlistingname}
\def\lstlistingautorefname{\langlistingautorefname}

% Abstände in Tabellen
\setlength{\tabcolsep}{\spaltenabstand}
\renewcommand{\arraystretch}{\zeilenabstand}
 

% Ab jetzt können auch Umlaute verwendet werden
\newcommand{\titel}{In der Regel haben wir einen zweizeiligen Bachelorthesistitel}
\newcommand{\martrikelnr}{1234567}
\newcommand{\kurs}{ABC2008DE}
\newcommand{\datumAbgabe}{August 2011}
\newcommand{\firma}{Firma GmbH}
\newcommand{\firmenort}{Firmenort}
\newcommand{\abgabeort}{Abgabeort}
\newcommand{\abschluss}{Bachelor of Engineering}
\newcommand{\studiengang}{Studienganges Vorderasiatische Archäologie}
\newcommand{\dhbw}{Stuttgart Campus Horb}
\newcommand{\betreuer}{Dipl.-Ing. (FH) Peter Pan}
\newcommand{\gutachter}{Dr. Silvana Koch-Mehrin}
\newcommand{\zeitraum}{12 Wochen}
\newcommand{\arbeitsart}{Bachelorarbeit}

\makeglossaries
%!TEX root = ../dokumentation.tex

%
% vorher in Konsole folgendes aufrufen: 
%	makeglossaries makeglossaries dokumentation.acn && makeglossaries dokumentation.glo
%

%
% Glossareintraege --> referenz, name, beschreibung
% Aufruf mit \gls{...}
%
\newglossaryentry{Glossareintrag}{name={Glossareintrag},plural={Glossareinträge},description={Ein Glossar beschreibt verschiedenste Dinge in kurzen Worten}}


\begin{document}

	% Deckblatt
	\begin{spacing}{1}
		\begin{titlepage}
	\begin{longtable}{p{.4\textwidth} p{.4\textwidth}}
	  {\includegraphics[height=2.6cm]{images/logo.png}} & 
	  {\includegraphics[height=2.6cm]{images/dhbw.png}}
	\end{longtable}
	\enlargethispage{20mm}
	\begin{center}
	  \vspace*{12mm}	{\LARGE\bf \titel }\\
	  \vspace*{12mm}	{\large\bf \arbeit}\\
	  \vspace*{12mm}	für die Prüfung zum\\
	  \vspace*{3mm} 	{\bf \abschluss}\\
	  \vspace*{12mm}	des \studiengang\\
	  \vspace*{3mm} 	an der Dualen Hochschule Baden-Württemberg \dhbw\\
	  \vspace*{12mm}	von\\
	  \vspace*{3mm} 	{\large\bf \autor}\\
	  \vspace*{12mm}	\datumAbgabe\\
	\end{center}
	\vfill
	\begin{spacing}{1.2}
	\begin{tabbing}
		mmmmmmmmmmmmmmmmmmmmmmmmmm     \= \kill
		\textbf{Bearbeitungszeitraum}  \>  \zeitraum\\
		\textbf{Matrikelnummer, Kurs}  \>  \martrikelnr, \kurs\\
		\textbf{Ausbildungsfirma}      \>  \firma, \firmenort\\
		\textbf{Betreuer}              \>  \betreuer\\
		\textbf{Gutachter}             \>  \gutachter
	\end{tabbing}
	\end{spacing}
\end{titlepage}

	\end{spacing}
	\newpage

	\renewcommand{\thepage}{\Roman{page}}
	\setcounter{page}{1}

	% Sperrvermerk
	%!TEX root = ../dokumentation.tex

\thispagestyle{empty}
% Sperrvermerk direkt hinter Titelseite
\section*{Sperrvermerk}

\vspace*{2em}

Die vorliegende {\arbeitsart} mit dem Titel {\itshape \titel} ist mit einem Sperrvermerk versehen und wird ausschließlich zu Prüfungszwecken am Studiengang {\studiengang} der Dualen Hochschule Baden-Württemberg {\abgabeort} vorgelegt.
Jede Einsichtnahme und Veröffentlichung – auch von Teilen der Arbeit – bedarf der vorherigen Zustimmung durch die {\firma}.

	\newpage
	
	% Erklärung
	%!TEX root = ../dokumentation.tex

\thispagestyle{empty}

\section*{Erklärung}
% http://www.se.dhbw-mannheim.de/fileadmin/ms/wi/dl_swm/dhbw-ma-wi-organisation-bewertung-bachelorarbeit-v2-00.pdf
\vspace*{2em}

Ich erkläre hiermit ehrenwörtlich: \\
\begin{enumerate}
\item dass ich meine {\arbeitsart} mit dem Thema
{\itshape \titel } ohne fremde Hilfe angefertigt habe;
\item dass ich die Übernahme wörtlicher Zitate aus der Literatur sowie die Verwendung der Gedanken
anderer Autoren an den entsprechenden Stellen innerhalb der Arbeit gekennzeichnet habe;
\item dass ich meine {\arbeitsart} bei keiner anderen Prüfung vorgelegt habe;
\item dass die eingereichte elektronische Fassung exakt mit der eingereichten schriftlichen Fassung
übereinstimmt.
\end{enumerate}

Ich bin mir bewusst, dass eine falsche Erklärung rechtliche Folgen haben wird.

\vspace{3em}

\abgabeort, \datumAbgabe
\vspace{4em}

\autor

	\newpage

	% Abstract
	%!TEX root = ../dokumentation.tex

\pagestyle{empty}

\iflang{de}{%
% Dieser deutsche Teil wird nur angezeigt, wenn die Sprache auf Deutsch eingestellt ist.
%Wenn das Abstract auf Deutsch sein soll die Zeile mit  \selectlanguage{english} auskommentieren
\selectlanguage{english}
\renewcommand{\abstractname}{Abstract}
\begin{abstract}
In englischer Sprache! Max. eine Seite und
eigenständig verständlich! 

Ein Abstract ist eine kurze und aussagekräftige Beschreibung des Inhalts der Arbeit. Der Umfang umfasst in
der Regel 200 bis 250 Wörter und beinhaltet die Fragestellung der Arbeit, die methodische Vorgehensweise
sowie die Hauptergebnisse der Arbeit. 

Quelle: \url{http://www.dhbw.de/fileadmin/user/public/Dokumente/Portal/Richtlinien_Praxismodule_Studien_und_Bachelorarbeiten_JG2011ff.pdf},Seite 22, Abgerufen 15.02.2015
\end{abstract}
\selectlanguage{ngerman}
}


\iflang{en}{%
% Dieser englische Teil wird nur angezeigt, wenn die Sprache auf Englisch eingestellt ist.
\begin{abstract}
An abstract is a brief summary of a research article, thesis, review,
conference proceeding or any in-depth analysis of a particular subject
or discipline, and is often used to help the reader quickly ascertain
the paper's purpose. When used, an abstract always appears at the
beginning of a manuscript, acting as the point-of-entry for any given
scientific paper or patent application. Abstracting and indexing
services for various academic disciplines are aimed at compiling a
body of literature for that particular subject.

The terms précis or synopsis are used in some publications to refer to
the same thing that other publications might call an "abstract". In
management reports, an executive summary usually contains more
information (and often more sensitive information) than the abstract
does.

Quelle: \url{http://en.wikipedia.org/wiki/Abstract_(summary)}

\end{abstract}
}
	\newpage

	\pagestyle{plain}

	% Inhaltsverzeichnis
	\begin{spacing}{1.1}
		\setcounter{tocdepth}{1}
		\tableofcontents
	\end{spacing}
	\newpage

	\renewcommand{\thepage}{\arabic{page}}
	\setcounter{page}{1}
	
	% Inhalt
	%!TEX root = ../dokumentation.tex

\chapter{Das erste Kapitel}
Hier steht z.\ B.\ eine kleine Einleitung...
\section{Verweise und Verlinkungen mit den Verzeichnissen}
\subsection{Akronyme / Abkürzungen}
Erste Erwähnung eines Akronyms wird ausgeschrieben mit Akürzung in Klammern angezeigt: \ac{AGPL}. Jede weitere wird nur in Kurzform verlinkt. Zweite Erwähnung: mehr zu \ac{AGPL} in \cite{fsf:2007}

\subsection{Verweise auf das Glossar}
Singular: \gls{Glossareintrag}, Plural: \glspl{Glossareintrag}

\subsection{Literaturverweise}
Nur erwähnte Literaturverweise werden auch im Literaturverzeichnis gedruckt:\\
Indirektes Zitat (vgl. \cite{baumgartner:2002}) oder auch ein ``direktes Zitat'', \cite{dreyfus:1980}.
\paragraph{}
\textquote[{\cite[S. 35]{baumgartner:2002}}]{Ein direktes Zitat, welches mehr als eine Zeile lang ist, sollte nicht im Fließtext stehen.}

\subsection{Bildverweise}
\begin{wrapfigure}{R}{.4\textwidth}
\includegraphics[height=.4\textwidth]{logo.png}
%\vspace{-20pt}
\caption{Musterfirmenlogo \cite{mustermann:2012}}
\label{fig:muster-logo1}
\end{wrapfigure}
In \autoref{fig:firmen-logo} auf Seite \pageref{fig:firmen-logo} könnte ein tatsächliches Firmenlogo zu sehen sein. Da dieser Text sehr kurz ist, rutscht \autoref{fig:muster-logo1} in den den folgenden Abschnitt, wo sie jedoch eigentlich nicht hingehört.


\section{Fließtext mit Abbildung}
Looking for the one superhero comic you just have to read. Following the antics and adventures of May Mayday Parker, this Spider-book has everything you could want in a comic--action, laughs, mystery and someone in a Spidey suit. Collects Alias \#1-28, What If. Jessica Jones had Joined the Avengers. In her inaugural arc, Jessicas life immediately becomes expendable when she uncovers the potentially explosive secret of one heros true identity. In her inaugural arc, Jessicas life immediately becomes expendable when she uncovers the potentially explosive secret of one heros true identity.

Once upon a time, Jessica Jones was a costumed super-hero, just not a very good one. First, a story where Wolverine and Hulk come together, and then Captain America and Cable meet up. In a city of Marvels, Jessica Jones never found her niche. The classic adventures of Spider-Man from the early days up until the 90s. Looking for the one superhero comic you just have to read. In her inaugural arc, Jessicas life immediately becomes expendable when she uncovers the potentially explosive secret of one heros true identity.

	%!TEX root = ../dokumentation.tex

\chapter{Beispiel Code-schnipsel einbinden}


%title wird unter dem Bsp. abgedruckt
%caption wird im Verzeichnis abgedruckt
%label wird zum referenzieren benutzt, muss einzigartig sein.
\lstset{caption=Code-Beispiel, label=Bsp.1, title=Hello World!,} 
\begin{lstlisting}
int main(void)
{
	system.out.printl{''Hello World!''}
	return 0;
} //Kommentar
\end{lstlisting}
%language ändert die Sprache, wenn nur eine Sprache verwendet wird diese Sprache gleich bei Einstellungen.tex ändern, standardmäßig Java
\lstset{caption=Pascal-Code, label=Pascal-Code, title=Pascal-Code,language=Pascal} 
\begin{lstlisting}
for i:=maxint to 0 do
begin
	{ do nothing }
end;
Write('Case insensitive ');
Write('Pascal keywords.');
\end{lstlisting}

\section{lorem ipsum}
Looking for the one superhero comic you just have to read. Following the antics and adventures of May Mayday Parker, this Spider-book has everything you could want in a comic--action, laughs, mystery and someone in a Spidey suit. Collects Alias\#1-28, What If. Jessica Jones had Joined the Avengers. In her inaugural arc, Jessicas life immediately becomes expendable when she uncovers the potentially explosive secret of one heros true identity.



Once upon a time, Jessica Jones was a costumed super-hero, just not a very good one. First, a story where Wolverine and Hulk come together, and then Captain America and Cable meet up. In a city of Marvels, Jessica Jones never found her niche. The classic adventures of Spider-Man from the early days up until the 90s. Looking for the one superhero comic you just have to read.



Meet all of Spideys deadly enemies, from the Green Goblin and Doctor Octopus to Venom and Carnage, plus see Peter Parker fall in love, face tragedy and triumph, and learn that with great power comes great responsibility. In a city of Marvels, Jessica Jones never found her niche. Bitten by a radioactive spider, high school student Peter Parker gained the speed, strength and powers of a spider. Looking for the one superhero comic you just have to read. What do you get when you ask the question, What if Spider-Man had a daughter.



The classic adventures of Spider-Man from the early days up until the 90s. Amazing Spider-Man is the cornerstone of the Marvel Universe. But will each partner’s combined strength be enough. Adopting the name Spider-Man, Peter hoped to start a career using his new abilities. Youve found it.

\section{Verweis auf Code}
Verweis auf den Code \ref{Bsp.1}.\\
Und der Pascal-Code \ref{Pascal-Code}.
	
	% Anhang
	\clearpage
	\pagenumbering{roman}

	% Abbildungsverzeichnis
	\cleardoublepage
	\phantomsection \label{listoffig}
	\addcontentsline{toc}{chapter}{Abbildungsverzeichnis}
	\listoffigures

	%Tabellenverzeichnis
	\cleardoublepage
	\phantomsection \label{listoftab}
	\addcontentsline{toc}{chapter}{Tabellenverzeichnis}
	\listoftables

	% Quellcodeverzeichnis
	\cleardoublepage
	\phantomsection \label{listoflist}
	\addcontentsline{toc}{chapter}{Listings}
	\lstlistoflistings

	% Literaturverzeichnis
	\cleardoublepage
	\phantomsection \label{listoflit}
	\addcontentsline{toc}{chapter}{Literaturverzeichnis}
	\begin{thebibliography}{---}

\bibitem[HAM10]{bib:ix042010}
  \textsc{Michael Hamm}: 
  \textbf{Der Erbe wächst - Freier Schwachstellen-Scanner OpenVAS 3.0}.
  iX Magazin, Ausgabe 4/2010, S. 81 ff., Heise Zeitschriften Verlag

\bibitem[NEU11]{bib:metasploitBuch}
  \textsc{Frank Neugebauer}: 
  \textbf{Penetration Testing mit Metasploit}.
  1. Auflage, 2011, dpunkt.verlag GmbH

\end{thebibliography}



	% Abkürzungsverzeichnis
	% vorher in Konsole folgendes aufrufen: 
	%	makeglossaries makeglossaries dokumentation.acn && makeglossaries dokumentation.glo
	\printglossary[type=\acronymtype]
	
	% Glossar
	\printglossary[style=altlist,title=Glossar]
\end{document}