%%**************************************************************
%% Vorlage fuer Bachelorarbeiten (o.ä.) der DHBW
%%
%% Autor: Tobias Dreher, Yves Fischer
%% Datum: 06.07.2011
%%
%% Autor: Michael Gruben
%% Datum: 15.05.2013
%%**************************************************************

%!TEX root = ../dokumentation.tex

%
% Nahezu alle Einstellungen koennen hier getaetigt werden
%

\documentclass[%
	pdftex,
	oneside,			% Einseitiger Druck.
	12pt,				% Schriftgroesse
	parskip=half,		% Halbe Zeile Abstand zwischen Absätzen.
	headsepline,		% Linie nach Kopfzeile.
	footsepline,		% Linie vor Fusszeile.
	abstracton,			% Abstract Überschriften
	listof=totoc,
	toc=bibliography,
]{scrreprt}

% Einstellungen laden
\usepackage{xstring}
\usepackage[utf8]{inputenc}
\usepackage[T1]{fontenc}

\newcommand{\einstellung}[1]{%
  \expandafter\newcommand\csname #1\endcsname{}
  \expandafter\newcommand\csname setze#1\endcsname[1]{\expandafter\renewcommand\csname#1\endcsname{##1}}
}
\newcommand{\langstr}[1]{\einstellung{lang#1}}

\einstellung{martrikelnr}
\einstellung{titel}
\einstellung{kurs}
\einstellung{datumAbgabe}
\einstellung{firma}
\einstellung{firmenort}
\einstellung{abgabeort}
\einstellung{abschluss}
\einstellung{studiengang}
\einstellung{dhbw}
\einstellung{betreuer}
\einstellung{gutachter}
\einstellung{zeitraum}
\einstellung{arbeit}
\einstellung{autor}
\einstellung{sprache}
\einstellung{schriftart}
\einstellung{seitenrand}
\einstellung{kapitelabstand}
\einstellung{spaltenabstand}
\einstellung{zeilenabstand}
\einstellung{zitierstil}
\einstellung{artikelArbeit}
\einstellung{meineArbeit}
 % verfügbare Einstellungen
%%%%%%%%%%%%%%%%%%%%%%%%%%%%%%%%%%%%%%%%%%%%%%%%%%%%%%%%%%%%%%%%%%%%%%%%%%%%%%%
%                                   Einstellungen
%
% Hier können alle relevanten Einstellungen für diese Arbeit gesetzt werden.
% Dazu gehören Angaben u.a. über den Autor sowie Formatierungen.
%
%
%%%%%%%%%%%%%%%%%%%%%%%%%%%%%%%%%%%%%%%%%%%%%%%%%%%%%%%%%%%%%%%%%%%%%%%%%%%%%%%


%%%%%%%%%%%%%%%%%%%%%%%%%%%%%%%%%%%% Sprache %%%%%%%%%%%%%%%%%%%%%%%%%%%%%%%%%%%
%% Aktuell sind Deutsch und Englisch unterstützt.
%% Es werden nicht nur alle vom Dokument erzeugten Texte in
%% der entsprechenden Sprache angezeigt, sondern auch weitere
%% Aspekte angepasst, wie z.B. die Anführungszeichen und
%% Datumsformate.
\setzesprache{de} % oder en
%%%%%%%%%%%%%%%%%%%%%%%%%%%%%%%%%%%%%%%%%%%%%%%%%%%%%%%%%%%%%%%%%%%%%%%%%%%%%%%%

%%%%%%%%%%%%%%%%%%%%%%%%%%%%%%%%%%% Angaben  %%%%%%%%%%%%%%%%%%%%%%%%%%%%%%%%%%%
%% Die meisten der folgenden Daten werden auf dem
%% Deckblatt angezeigt, einige auch im weiteren Verlauf
%% des Dokuments.
\setzemartrikelnr{1234510}
\setzekurs{ABC2008DE}
\setzetitel{In der Regel haben wir einen zweizeiligen Bachelorthesistitel}
\setzedatumAbgabe{August 2011}
\setzefirma{Firma GmbH}
\setzefirmenort{Firmenort}
\setzeabgabeort{Abgabeort}
\setzeabschluss{Bachelor of Engineering}
\setzestudiengang{Vorderasiatische Archäologie}
\setzedhbw{Karlsruhe}
\setzebetreuer{Dipl.-Ing.~(FH) Peter Pan}
\setzegutachter{Dr.\ Silvana Koch-Mehrin}
\setzezeitraum{12 Wochen}
\setzearbeit{Bachelorarbeit}
\setzeautor{Vorname Nachname}
%nur bei Sprachauswahl de verwendet
\setzeartikelArbeit{Die} %bestimmt den Artikel der Arbeit, Bsp. Die Bachelorarbeit, Der Projektbericht
\setzemeineArbeit{meine} % ...,dass ich meine/n Arbeit ...  Bsp. meine Bachelorarbeit, meinen Projektbericht
%%%%%%%%%%%%%%%%%%%%%%%%%%%%%%%%%%%%%%%%%%%%%%%%%%%%%%%%%%%%%%%%%%%%%%%%%%%%%%%%

%%%%%%%%%%%%%%%%%%%%%%%%%%%% Literaturverzeichnis %%%%%%%%%%%%%%%%%%%%%%%%%%%%%%
%% Bei Fehlern während der Verarbeitung bitte in ads/header.tex bei der
%% Einbindung des Pakets biblatex (ungefähr ab Zeile 110,
%% einmal für jede Sprache), biber in bibtex ändern.
\newcommand{\ladeliteratur}{%
\addbibresource{bibliographie.bib}
%\addbibresource{weitereDatei.bib}
}
%% Zitierstil
%% siehe: http://ctan.mirrorcatalogs.com/macros/latex/contrib/biblatex/doc/biblatex.pdf (3.3.1 Citation Styles)
%% mögliche Werte z.B numeric-comp, alphabetic, authoryear
\setzezitierstil{numeric-comp}
%%%%%%%%%%%%%%%%%%%%%%%%%%%%%%%%%%%%%%%%%%%%%%%%%%%%%%%%%%%%%%%%%%%%%%%%%%%%%%%%

%%%%%%%%%%%%%%%%%%%%%%%%%%%%%%%%% Layout %%%%%%%%%%%%%%%%%%%%%%%%%%%%%%%%%%%%%%%
%% Verschiedene Schriftarten
% laut nag Warnung: palatino obsolete, use mathpazo, helvet (option scaled=.95), courier instead
\setzeschriftart{lmodern} % palatino oder goudysans, lmodern, libertine

%% Paket um Textteile drehen zu können
%\usepackage{rotating}
%% Paket um Seite im Querformat anzuzeigen
%\usepackage{lscape}

%% Seitenränder
\setzeseitenrand{2.5cm}

%% Abstand vor Kapitelüberschriften zum oberen Seitenrand
\setzekapitelabstand{20pt}

%% Spaltenabstand
\setzespaltenabstand{10pt}
%%Zeilenabstand innerhalb einer Tabelle
\setzezeilenabstand{1.5}
%%%%%%%%%%%%%%%%%%%%%%%%%%%%%%%%%%%%%%%%%%%%%%%%%%%%%%%%%%%%%%%%%%%%%%%%%%%%%%%%

%%%%%%%%%%%%%%%%%%%%%%%%%%%%% Verschiedenes  %%%%%%%%%%%%%%%%%%%%%%%%%%%%%%%%%%%
%% Farben (Angabe in HTML-Notation mit großen Buchstaben)
\newcommand{\ladefarben}{%
	\definecolor{LinkColor}{HTML}{00007A}
	\definecolor{ListingBackground}{HTML}{FCF7DE}
}
%% Mathematikpakete benutzen (Pakete aktivieren)
%\usepackage{amsmath}
%\usepackage{amssymb}

%% Programmiersprachen Highlighting (Listings)
\newcommand{\listingsettings}{%
	\lstset{%
		language=Java,			% Standardsprache des Quellcodes
		numbers=left,			% Zeilennummern links
		stepnumber=1,			% Jede Zeile nummerieren.
		numbersep=5pt,			% 5pt Abstand zum Quellcode
		numberstyle=\tiny,		% Zeichengrösse 'tiny' für die Nummern.
		breaklines=true,		% Zeilen umbrechen wenn notwendig.
		breakautoindent=true,	% Nach dem Zeilenumbruch Zeile einrücken.
		postbreak=\space,		% Bei Leerzeichen umbrechen.
		tabsize=2,				% Tabulatorgrösse 2
		basicstyle=\ttfamily\footnotesize, % Nichtproportionale Schrift, klein für den Quellcode
		showspaces=false,		% Leerzeichen nicht anzeigen.
		showstringspaces=false,	% Leerzeichen auch in Strings ('') nicht anzeigen.
		extendedchars=true,		% Alle Zeichen vom Latin1 Zeichensatz anzeigen.
		captionpos=b,			% sets the caption-position to bottom
		backgroundcolor=\color{ListingBackground}, % Hintergrundfarbe des Quellcodes setzen.
		xleftmargin=0pt,		% Rand links
		xrightmargin=0pt,		% Rand rechts
		frame=single,			% Rahmen an
		frameround=ffff,
		rulecolor=\color{darkgray},	% Rahmenfarbe
		fillcolor=\color{ListingBackground},
		keywordstyle=\color[rgb]{0.133,0.133,0.6},
		commentstyle=\color[rgb]{0.133,0.545,0.133},
		stringstyle=\color[rgb]{0.627,0.126,0.941}
	}
}
%%%%%%%%%%%%%%%%%%%%%%%%%%%%%%%%%%%%%%%%%%%%%%%%%%%%%%%%%%%%%%%%%%%%%%%%%%%%%%%%

%%%%%%%%%%%%%%%%%%%%%%%%%%%%%%%% Eigenes %%%%%%%%%%%%%%%%%%%%%%%%%%%%%%%%%%%%%%%
%% Hier können Ergänzungen zur Präambel vorgenommen werden (eigene Pakete, Einstellungen)


\usepackage{pdfpages}
 % lese Einstellungen

\newcommand{\iflang}[2]{%
  \IfStrEq{\sprache}{#1}{#2}{}
}

\langstr{abkverz}
\langstr{glossar}
\langstr{deckblattabschlusshinleitung}
\langstr{artikelstudiengang}
\langstr{studiengang}
\langstr{anderdh}
\langstr{von}
\langstr{dbbearbeitungszeit}
\langstr{dbmatriknr}
\langstr{dbkurs}
\langstr{dbfirma}
\langstr{dbbetreuer}
\langstr{dbgutachter}
\langstr{sperrvermerk}
\langstr{erklaerung}
\langstr{listingname}
\langstr{listlistingname}
\langstr{listingautorefname} % verfügbare Strings
\input{lang/\sprache} % Übersetzung einlesen

\usepackage[english, ngerman]{babel}
\iflang{de}{\selectlanguage{ngerman}} % Paket babel benutzt neue deutsche Rechtschreibung
\iflang{en}{\selectlanguage{english}} % Paket babel benutzt Englisch


%%%%%%% Package Includes %%%%%%%

\usepackage[margin=1in]{geometry}
\usepackage[activate]{microtype} %Zeilenumbruch und mehr
\usepackage[onehalfspacing]{setspace}
\usepackage{makeidx}
\usepackage[autostyle=true,german=quotes]{csquotes}
\usepackage{longtable}
\usepackage{graphicx}
\usepackage{xcolor} 	%xcolor für HTML-Notation
\usepackage{float}
\usepackage{array}
\usepackage{calc}		%zum Rechnen (Bildtabelle in Deckblatt)
\usepackage[right]{eurosym}
\usepackage{wrapfig}
\usepackage{pgffor} % für automatische Kapiteldateieinbindung
\usepackage[perpage, hang, multiple, stable]{footmisc}
\usepackage[printonlyused,footnote]{acronym}
\usepackage[nonumberlist,toc]{glossaries}
\usepackage{listings}

% eine Kommentarumgebung "k" (Handhabe mit \begin{k}<Kommentartext>\end{k},
% Kommentare werden rot gedruckt). Wird \% vor excludecomment{k} entfernt,
% werden keine Kommentare mehr gedruckt.
\usepackage{comment}
\specialcomment{k}{\begingroup\color{red}}{\endgroup}
%\excludecomment{k}


%%%%%% Configuration %%%%%

%% Anwenden der Einstellungen

\usepackage{\schriftart}
\ladefarben{}

% Titel, Autor und Datum
\title{\titel}
\author{\autor}
\date{\datum}

% PDF Einstellungen
\usepackage[%
	pdftitle={\titel},
	pdfauthor={\autor},
	pdfsubject={\arbeit},
	pdfcreator={pdflatex, LaTeX with KOMA-Script},
	pdfpagemode=UseOutlines, 		% Beim Oeffnen Inhaltsverzeichnis anzeigen
	pdfdisplaydoctitle=true, 		% Dokumenttitel statt Dateiname anzeigen.
	pdflang={\sprache}, 			% Sprache des Dokuments.
]{hyperref}

% (Farb-)einstellungen für die Links im PDF
\hypersetup{%
	colorlinks=true, 		% Aktivieren von farbigen Links im Dokument
	linkcolor=LinkColor, 	% Farbe festlegen
	citecolor=LinkColor,
	filecolor=LinkColor,
	menucolor=LinkColor,
	urlcolor=LinkColor,
	linktocpage=true, 		% Nicht der Text sondern die Seitenzahlen in Verzeichnissen klickbar
	bookmarksnumbered=true 	% Überschriftsnummerierung im PDF Inhalt anzeigen.
}
% Workaround um Fehler in Hyperref, muss hier stehen bleiben
\usepackage{bookmark} %nur ein latex-Durchlauf für die Aktualisierung von Verzeichnissen nötig

% Schriftart in Captions etwas kleiner
\addtokomafont{caption}{\small}

% Literaturverweise (sowohl deutsch als auch englisch)
\iflang{de}{%
\usepackage[
	backend=biber,		% empfohlen. Falls biber Probleme macht: bibtex
	bibwarn=true,
	bibencoding=utf8,	% wenn .bib in utf8, sonst ascii
	sortlocale=de_DE,
	style=alphabetic	%Zitierstil. Siehe http://ctan.mirrorcatalogs.com/macros/latex/contrib/biblatex/doc/biblatex.pdf
]{biblatex}
}
\iflang{en}{%
\usepackage[
	backend=biber,		% empfohlen. Falls biber Probleme macht: bibtex
	bibwarn=true,
	bibencoding=utf8,	% wenn .bib in utf8, sonst ascii
	sortlocale=en_US,
	style=alphabetic	%Zitierstil. Siehe http://ctan.mirrorcatalogs.com/macros/latex/contrib/biblatex/doc/biblatex.pdf
]{biblatex}
}

\ladeliteratur{}

%%%%%% Additional settings %%%%%%

% Hurenkinder und Schusterjungen verhindern
% http://projekte.dante.de/DanteFAQ/Silbentrennung
\clubpenalty=10000
\widowpenalty=10000
\displaywidowpenalty=10000

% Bildpfad
\graphicspath{{images/}}

% Einige häufig verwendete Sprachen
\lstloadlanguages{PHP,Python,Java,C,C++,bash}
\listingsettings{}

\setlength{\tabcolsep}{\spaltenabstand}
\renewcommand{\arraystretch}{\zeilenabstand}


%Umbennung der listingsnamen
\renewcommand\lstlistingname{\langlistingname}
\renewcommand\lstlistlistingname{\langlistlistingname}
\def\lstlistingautorefname{\langlistingautorefname}


% Ab jetzt können auch Umlaute verwendet werden

%falls pdftitel = titel der Arbeit
\newcommand{\titel}{\pdftitel}
%bei unterschiedlichen Titeln
%\newcommand{\titel}{In der Regel haben wir einen zweizeiligen
% Bachelorthesistitel}
\newcommand{\martrikelnr}{1234567}
\newcommand{\kurs}{ABC2008DE}
\newcommand{\datumAbgabe}{August 2011}
\newcommand{\firma}{Firma GmbH}
\newcommand{\firmenort}{Firmenort}
\newcommand{\abgabeort}{Abgabeort}
\newcommand{\abschluss}{Bachelor of Engineering}
\newcommand{\studiengang}{Studienganges Vorderasiatische Archäologie}
\newcommand{\dhbw}{Stuttgart Campus Horb}
\newcommand{\betreuer}{Dipl.-Ing. (FH) Peter Pan}
\newcommand{\gutachter}{Dr. Silvana Koch-Mehrin}
\newcommand{\zeitraum}{12 Wochen}
\newcommand{\arbeitsart}{\arbeit}

\makeglossaries
%!TEX root = ../dokumentation.tex

%
% vorher in Konsole folgendes aufrufen: 
%	makeglossaries makeglossaries dokumentation.acn && makeglossaries dokumentation.glo
%

%
% Glossareintraege --> referenz, name, beschreibung
% Aufruf mit \gls{...}
%
\newglossaryentry{Glossareintrag}{name={Glossareintrag},plural={Glossareinträge},description={Ein Glossar beschreibt verschiedenste Dinge in kurzen Worten}}


\begin{document}

	% Deckblatt
	\begin{spacing}{1}
		%!TEX root = ../dokumentation.tex

\begin{titlepage}
	\begin{longtable}{p{8.2cm} p{5.4cm}}
		{\raisebox{\ht\strutbox-\totalheight}{\includegraphics{images/firma-deckblatt.png}}} &
		{\raisebox{\ht\strutbox-\totalheight}{\includegraphics[height=2.5cm]{images/dhbw.png}}}
	\end{longtable}
	\enlargethispage{20mm}
	\begin{center}
		\vspace*{12mm}	{\LARGE\textbf \titel }\\
		\vspace*{12mm}	{\large\textbf \arbeit}\\
		\vspace*{12mm}	\langdeckblattabschlusshinleitung\\
		\vspace*{3mm}		{\textbf \abschluss}\\
		\vspace*{12mm}	\langartikelstudiengang{} \langstudiengang{} \studiengang\\
    \vspace*{3mm}		\langanderdh{} \dhbw\\
		\vspace*{12mm}	\langvon\\
		\vspace*{3mm}		{\large\textbf \autor}\\
		\vspace*{12mm}	\datumAbgabe\\
	\end{center}
	\vfill
	\begin{spacing}{1.2}
	\begin{tabbing}
		mmmmmmmmmmmmmmmmmmmmmmmmmm             \= \kill
		\textbf{\langdbbearbeitungszeit}       \>  \zeitraum\\
		\textbf{\langdbmatriknr, \langdbkurs}  \>  \martrikelnr, \kurs\\
		\textbf{\langdbfirma}                  \>  \firma, \firmenort\\
		\textbf{\langdbbetreuer}               \>  \betreuer\\
		\textbf{\langdbgutachter}              \>  \gutachter
	\end{tabbing}
	\end{spacing}
\end{titlepage}

	\end{spacing}
	\newpage

	\renewcommand{\thepage}{\Roman{page}}
	\setcounter{page}{1}

	% Sperrvermerk
	%!TEX root = ../dokumentation.tex

\thispagestyle{empty}
% Sperrvermerk direkt hinter Titelseite
\section*{Sperrvermerk}

\vspace*{2em}

Die vorliegende {\arbeitsart} mit dem Titel {\itshape \titel} ist mit einem Sperrvermerk versehen und wird ausschließlich zu Prüfungszwecken am Studiengang {\studiengang} der Dualen Hochschule Baden-Württemberg {\abgabeort} vorgelegt.
Jede Einsichtnahme und Veröffentlichung – auch von Teilen der Arbeit – bedarf der vorherigen Zustimmung durch die {\firma}.

	\newpage
	
	% Erklärung
	%!TEX root = ../dokumentation.tex

\thispagestyle{empty}

\section*{\langerklaerung}
% http://www.se.dhbw-mannheim.de/fileadmin/ms/wi/dl_swm/dhbw-ma-wi-organisation-bewertung-bachelorarbeit-v2-00.pdf
\vspace*{2em}

\iflang{de}{%
Ich erkläre hiermit ehrenwörtlich: \\
\begin{enumerate}
\item dass ich {\meineArbeit} {\arbeit} mit dem Thema
{\itshape \titel } ohne fremde Hilfe angefertigt habe;
\item dass ich die Übernahme wörtlicher Zitate aus der Literatur sowie die Verwendung der Gedanken
anderer Autoren an den entsprechenden Stellen innerhalb der Arbeit gekennzeichnet habe;
\item dass ich {\meineArbeit} {\arbeit} bei keiner anderen Prüfung vorgelegt habe;
\item dass die eingereichte elektronische Fassung exakt mit der eingereichten schriftlichen Fassung
übereinstimmt.
\end{enumerate}

Ich bin mir bewusst, dass eine falsche Erklärung rechtliche Folgen haben wird.
}

% http://www.ib.dhbw-mannheim.de/fileadmin/ms/bwl-ib/Downloads_alt/Leitfaden_31.05.pdf (S. 52)

\iflang{en}{%
Hereby I solemnly declare:
\begin{enumerate}
\item that this {\arbeit}, titled {\itshape \titel } is entirely the product of my own scholarly work, unless otherwise indicated in the text or references, or acknowledged below;
\item I have indicated the thoughts adopted directly or indirectly from other sources at the appropriate places within the document;
\item this {\arbeit} has not been submitted either in whole or part, for a degree at this or any other university or institution;
\item I have not published this {\arbeit} in the past; 
\item the printed version is equivalent to the submitted electronic one.
\end{enumerate}
I am aware that a dishonest declaration will entail legal consequences.
}

\vspace{3em}

\abgabeort, \datumAbgabe
\vspace{4em}

\rule{6cm}{0.4pt}\\
\autor

	\newpage

	% Abstract
	%!TEX root = ../dokumentation.tex

\pagestyle{empty}

\iflang{de}{%
% Dieser deutsche Teil wird nur angezeigt, wenn die Sprache auf Deutsch eingestellt ist.
\renewcommand{\abstractname}{Abstract} % Text für Überschrift
\begin{abstract}
Ein Abstract ist eine prägnante Inhaltsangabe, ein Abriss ohne
Interpretation und Wertung einer wissenschaftlichen Arbeit. In DIN
1426 wird das (oder auch der) Abstract als Kurzreferat zur
Inhaltsangabe beschrieben.

\begin{description}
\item[Objektivität] soll sich jeder persönlichen Wertung enthalten
\item[Kürze] soll so kurz wie möglich sein
\item[Genauigkeit] soll genau die Inhalte und die Meinung der Originalarbeit wiedergeben
\end{description}

Üblicherweise müssen wissenschaftliche Artikel einen Abstract
enthalten, typischerweise von 100-150 Wörtern, ohne Bilder und
Literaturzitate und in einem Absatz.

Quelle \url{http://de.wikipedia.org/wiki/Abstract} Abgerufen 07.07.2011
\end{abstract}
}


\iflang{en}{%
% Dieser englische Teil wird nur angezeigt, wenn die Sprache auf Englisch eingestellt ist.
\renewcommand{\abstractname}{Abstract} % Text für Überschrift
\begin{abstract}
An abstract is a brief summary of a research article, thesis, review,
conference proceeding or any in-depth analysis of a particular subject
or discipline, and is often used to help the reader quickly ascertain
the paper's purpose. When used, an abstract always appears at the
beginning of a manuscript, acting as the point-of-entry for any given
scientific paper or patent application. Abstracting and indexing
services for various academic disciplines are aimed at compiling a
body of literature for that particular subject.

The terms précis or synopsis are used in some publications to refer to
the same thing that other publications might call an "abstract". In
management reports, an executive summary usually contains more
information (and often more sensitive information) than the abstract
does.

Quelle: \url{http://en.wikipedia.org/wiki/Abstract_(summary)}

\end{abstract}
}
	\newpage

	\pagestyle{plain}

	% Inhaltsverzeichnis
	\begin{spacing}{1.1}
		\setcounter{tocdepth}{1}
		%für die Anzeige von Unterkapiteln im Inhaltsverzeichnis
		%\setcounter{tocdepth}{2}
		\tableofcontents
	\end{spacing}
	\newpage

	\renewcommand{\thepage}{\arabic{page}}
	\setcounter{page}{1}
	
	% Inhalt
	%!TEX root = ../dokumentation.tex

\chapter{Das erste Kapitel}
Erste Erwähnung eines Akronyms wird als Fußnote angezeigt. Jede weitere wird
nur verlinkt: \ac{AGPL}. Zweite Erwähnung \ac{AGPL}

Verweise auf das Glossar: \gls{Glossareintrag}, \glspl{Glossareintrag}

Nur erwähnte Literaturverweise werden auch im Literaturverzeichnis gedruckt:
\cite{baumgartner:2002}, \cite{dreyfus:1980}

Meine erste Fußnote\footnote{Ich bin eine Fußnote}

\begin{wrapfigure}{r}{.4\textwidth}
\includegraphics[height=.4\textwidth]{logo.png}
\vspace{-15pt}
\caption{Das Logo der Musterfirma\footnotemark}
\end{wrapfigure}
%Quelle muss in Fußnote stehen (da sonst aufgrund eines Fehlers nicht kompiliert
% wird)
\footnotetext{aus \cite{mustermann:2012}}
Ein ganz langer Text, der das Bild umfließt. Ein ganz langer Text, der das Bild
umfließt. Ein ganz langer Text, der das Bild umfließt. Ein ganz langer Text, der
das Bild umfließt. Ein ganz langer Text, der das Bild umfließt. Ein ganz langer
Text, der das Bild umfließt. Ein ganz langer Text, der das Bild umfließt. Ein
ganz langer Text, der das Bild umfließt. Ein ganz langer Text, der das Bild
umfließt. Ein ganz langer Text, der das Bild umfließt. Ein ganz langer Text, der
das Bild umfließt. Ein ganz langer Text, der das Bild umfließt. Ein ganz langer Text, der das Bild
umfließt. Ein ganz langer Text, der das Bild umfließt. Ein ganz langer Text, der
das Bild umfließt. Ein ganz langer Text, der das Bild umfließt. Ein ganz langer
Text, der das Bild umfließt. Ein ganz langer Text, der das Bild umfließt. Ein
ganz langer Text, der das Bild umfließt. Ein ganz langer Text, der das Bild
umfließt. Ein ganz langer Text, der das Bild umfließt. Ein ganz langer Text, der
das Bild umfließt.

	%!TEX root = ../dokumentation.tex

\chapter{Beispiel Code-schnipsel einbinden}


%title wird unter dem Bsp. abgedruckt
%caption wird im Verzeichnis abgedruckt
%label wird zum referenzieren benutzt, muss einzigartig sein.
\lstset{caption=Code-Beispiel, label=Bsp.1, title=Hello World!,} 
\begin{lstlisting}
int main(void)
{
	system.out.printl{''Hello World!''}
	return 0;
} //Kommentar
\end{lstlisting}
%language ändert die Sprache, wenn nur eine Sprache verwendet wird diese Sprache gleich bei Einstellungen.tex ändern, standardmäßig Java
\lstset{caption=Pascal-Code, label=Pascal-Code, title=Pascal-Code,language=Pascal} 
\begin{lstlisting}
for i:=maxint to 0 do
begin
	{ do nothing }
end;
Write('Case insensitive ');
Write('Pascal keywords.');
\end{lstlisting}

\section{lorem ipsum}
Looking for the one superhero comic you just have to read. Following the antics and adventures of May Mayday Parker, this Spider-book has everything you could want in a comic--action, laughs, mystery and someone in a Spidey suit. Collects Alias\#1-28, What If. Jessica Jones had Joined the Avengers. In her inaugural arc, Jessicas life immediately becomes expendable when she uncovers the potentially explosive secret of one heros true identity.



Once upon a time, Jessica Jones was a costumed super-hero, just not a very good one. First, a story where Wolverine and Hulk come together, and then Captain America and Cable meet up. In a city of Marvels, Jessica Jones never found her niche. The classic adventures of Spider-Man from the early days up until the 90s. Looking for the one superhero comic you just have to read.



Meet all of Spideys deadly enemies, from the Green Goblin and Doctor Octopus to Venom and Carnage, plus see Peter Parker fall in love, face tragedy and triumph, and learn that with great power comes great responsibility. In a city of Marvels, Jessica Jones never found her niche. Bitten by a radioactive spider, high school student Peter Parker gained the speed, strength and powers of a spider. Looking for the one superhero comic you just have to read. What do you get when you ask the question, What if Spider-Man had a daughter.



The classic adventures of Spider-Man from the early days up until the 90s. Amazing Spider-Man is the cornerstone of the Marvel Universe. But will each partner’s combined strength be enough. Adopting the name Spider-Man, Peter hoped to start a career using his new abilities. Youve found it.

\section{Verweis auf Code}
Verweis auf den Code \ref{Bsp.1}.\\
Und der Pascal-Code \ref{Pascal-Code}.
	
	% Anhang
	\clearpage
	\pagenumbering{roman}

	% Abbildungsverzeichnis
	\cleardoublepage
	\phantomsection \label{listoffig}
	\addcontentsline{toc}{chapter}{Abbildungsverzeichnis}
	\listoffigures

	%Tabellenverzeichnis
	\cleardoublepage
	\phantomsection \label{listoftab}
	\addcontentsline{toc}{chapter}{Tabellenverzeichnis}
	\listoftables

	% Quellcodeverzeichnis
	\cleardoublepage
	\phantomsection \label{listoflist}
	\addcontentsline{toc}{chapter}{Listings}
	\lstlistoflistings

	% Literaturverzeichnis
	\cleardoublepage
	\phantomsection \label{listoflit}
	\addcontentsline{toc}{chapter}{Literaturverzeichnis}
	
	%Harvard
	\bibliographystyle{agsm}
	
	%Durchnummeriert
	%\bibliographystyle{ams}
	
	%Kürzel für Autor und Jahr
	%\bibliographystyle{amsalpha}
	%%%%%%%alle Stile in
	% http://amath.colorado.edu/documentation/LaTeX/reference/faq/bibstyles.pdf
	
	\bibliography{ArbeitBib}

	% Abkürzungsverzeichnis
	\cleardoublepage
	\phantomsection \label{listofacs}
	\addcontentsline{toc}{chapter}{Abkürzungsverzeichnis}
	\chapter*{Abkürzungsverzeichnis}
%nur verwendete Akronyme werden letztlich im Dokument angezeigt
\begin{acronym}[YTMMM]
\setlength{\itemsep}{-\parsep}

\acro{AGPL}{Affero GNU General Public License}
\acro{API}{Application Programming Interface}
\acro{WYSIWYG}{What You See Is What You Get}
\end{acronym}
	
	% Glossar
	\printglossary[style=altlist,title=Glossar]
\end{document}